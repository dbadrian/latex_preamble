%%%%%%%%%%%%%%%%%%%%%%%%%%%%%%%%%%%%%%%%%%%%%%%%%%%%%%%%%%%%%%%%%%%%%%%%
% General
%%%%%%%%%%%%%%%%%%%%%%%%%%%%%%%%%%%%%%%%%%%%%%%%%%%%%%%%%%%%%%%%%%%%%%%
%\newcommand\todo[1]{\textcolor{red}{TODO: #1}} % Marks the arg red and prepends TODO

\newcommand{\arrowright}{$\rightarrow$} % Use math error in text

%%%%%%%%%%%%%%%%%%%%%%%%%%%%%%%%%%%%%%%%%%%%%%%%%%%%%%%%%%%%%%%%%%%%%%%%
% Theorms etc.
%%%%%%%%%%%%%%%%%%%%%%%%%%%%%%%%%%%%%%%%%%%%%%%%%%%%%%%%%%%%%%%%%%%%%%%%
%\newtheorem{theorem}{Theorem}[section]
%\newtheorem{corollary}{Corollary}[theorem]
%\newtheorem{lemma}[theorem]{Lemma}
%\newtheorem{hypothesis}{Hypothesis}[section]
%\newtheorem{hypoexp}{Experiment}[hypothesis]


%%%%%%%%%%%%%%%%%%%%%%%%%%%%%%%%%%%%%%%%%%%%%%%%%%%%%%%%%%%%%%%%%%%%%%%%
% Acro / Abbreviation
%%%%%%%%%%%%%%%%%%%%%%%%%%%%%%%%%%%%%%%%%%%%%%%%%%%%%%%%%%%%%%%%%%%%%%%%
% Useful for using AC in UPPERCASE titles in e.g., Beamer template
%\newcommand{\AC}[1]{\ac[long-format=\MakeUppercase]{#1}}
%\newcommand{\ACP}[1]{\acp[long-format=\MakeUppercase]{#1}}

%%%%%%%%%%%%%%%%%%%%%%%%%%%%%%%%%%%%%%%%%%%%%%%%%%%%%%%%%%%%%%%%%%%%%%%%
% CS
%%%%%%%%%%%%%%%%%%%%%%%%%%%%%%%%%%%%%%%%%%%%%%%%%%%%%%%%%%%%%%%%%%%%%%%%
% complexity
\newcommand{\bigo}[1]{\mathcal{O}(#1)}  % Big-O notation

%%%%%%%%%%%%%%%%%%%%%%%%%%%%%%%%%%%%%%%%%%%%%%%%%%%%%%%%%%%%%%%%%%%%%%%%
% Machine Learning
%%%%%%%%%%%%%%%%%%%%%%%%%%%%%%%%%%%%%%%%%%%%%%%%%%%%%%%%%%%%%%%%%%%%%%%%
\newcommand{\train}{\mathcal{D_{\mathrm{train}}}}
\newcommand{\valid}{\mathcal{D_{\mathrm{val}}}}
\newcommand{\test}{\mathcal{D_{\mathrm{test}}}}


%%%%%%%%%%%%%%%%%%%%%%%%%%%%%%%%%%%%%%%%%%%%%%%%%%%%%%%%%%%%%%%%%%%%%%%%
% Math
%%%%%%%%%%%%%%%%%%%%%%%%%%%%%%%%%%%%%%%%%%%%%%%%%%%%%%%%%%%%%%%%%%%%%%%%
% Placeholder dots with improved styling/shortcut name
\newcommand*{\tdot}{\makebox[1ex]{\textbf{$\cdot$}}}%
\newcommand*{\bdot}{\makebox[1ex]{\textbf{$\bullet$}}}%

% Math shortcuts
\newcommand{\mr}{\mathrm}
\newcommand{\ms}{\mathsf}
\newcommand{\mc}{\mathcal}
%\newcommand{\wt}[1]{{\widetilde{#1}}}

%% Vectors, Matrix, Tensors, Arrays
\newcommand{\vc}[1]{\mathbfit{#1}} % A vector
\newcommand{\mat}[1]{\mathbfit{#1}} % A matrix
\newcommand{\ten}[1]{\mathbf{#1}} % A tensor

%% Special vector-types
\newcommand{\vct}[1]{\bm{ \mathfrak{#1}}} % A vector with temporal dependency
\newcommand{\uv}[1]{\hat{\vc{#1}}} % a unit vector

%% Special Vectors, Matrices, etc.
\newcommand{\sbv}[1]{\vc{e}_{#1}} % standard basis vectors

\newcommand{\matI}{\mat I}
\newcommand{\matJ}{\mat J} % same as an all ones matrix
\newcommand{\matO}{\mat 0}
\newcommand{\vecO}{\matO}
\newcommand{\vecI}{\mathbfsf{1}}

\newcommand{\eye}[1]{\matI_{#1}}
\newcommand{\zeros}[1]{\matO_{#1}}
\newcommand{\ones}[1]{\mat{1}_{#1}}
\newcommand{\onesJ}[1]{\matJ_{#1}}

%% Sets and Graphs
%\newcommand{\set}[1]{#1}  % set
\newcommand{\graph}[1]{\mathcal{#1}} % graph
\newcommand{\graphdef}[1]{\graph{#1}=(\set{V},\set{E})}  % standard graph def
\DeclarePairedDelimiter{\card}{\vert}{{\vert}} % Cardinality of set


%% Special Sets
\newcommand{\R}{\mathbb{R}}  % set of real numbers 
\newcommand{\N}{\mathbb{N}} % set of natural numbers
%\newcommand{\Z}{\mathbb{Z}} % set of integers
\newcommand{\Q}{\mathbb{Q}} % set of rational numbers
\newcommand{\A}{\mathbb{A}} % set of algebraic numbers
\newcommand{\C}{\mathbb{C}} % set of complex numbers
\renewcommand{\H}{\mathbb{H}} % set of quaternions
% \newcommand{\bspc}[1]{\{0,1\}^{#1}}

%% Symbols
\def\eps{{\epsilon}}

%% LA Operations
\DeclarePairedDelimiter{\prt}{\left(}{\right)} % parantheses helper
%\newcommand{\iprod}[2]{\langle #1, #2 \rangle} % inner product
%\DeclarePairedDelimiter{\norm}{\lVert}{\rVert} % norm
\newcommand{\pnorm}[2]{\norm{#1}_{#2}} % p-norm (takes arguent)
%\DeclarePairedDelimiter{\ceil}{\lceil}{\rceil} % ceil
%\DeclarePairedDelimiter{\floor}{\lfloor}{\rfloor} % floor
\newcommand{\stack}[2]{[#1;#2]} % stacking of vectors
\DeclareMathOperator{\vecc}{\mathrm{vec}}
\DeclareMathOperator{\diag}{\mathrm{diag}}
\newcommand{\todiag}[1]{\diag\prt{#1}} % to diagonal matrix op
\newcommand{\tovec}[1]{\vecc\prt{#1}} % to vector op
\DeclareMathOperator{\Tr}{Tr} % trace
\newcommand{\Trp}[1]{\Tr\prt{#1}} % trace
\newcommand{\detp}[1]{\det\prt{#1}} % determinant
%\DeclarePairedDelimiter\abs{\lvert}{\rvert}%

% % Swap the definition of \abs* and \norm*, so that \abs
% % and \norm resizes the size of the brackets, and the 
% % starred version does not.
% \makeatletter
% \let\oldabs\abs
% \def\abs{\@ifstar{\oldabs}{\oldabs*}}
% %
% \let\oldnorm\norm
% \def\norm{\@ifstar{\oldnorm}{\oldnorm*}}
% \makeatother

\newcommand{\expnmb}[2]{{#1}\mathrm{e}{#2}}

\DeclareMathOperator*{\argmax}{arg\,max}
\DeclareMathOperator*{\argmin}{arg\,min}

\DeclareMathOperator{\sign}{sgn}
\newcommand{\signp}[1]{\sign\prt{#1}}

\DeclareMathOperator{\MSE}{MSE}


%% Shortcuts for denoting spaces in practise
\newcommand{\spc}[2]{\mathbb{#1}^{#2}}
\newcommand{\rspc}[1]{\R^{#1}}


\newcommand{\cmark}{\ding{51}}
\newcommand{\xmark}{\ding{55}}
% extra math cmds
%\newcommand{\norm}[1]{\left\lVert#1\right\rVert}

%\newcommand{\img}[2]{\mat{I}_{#1}^{#2}}
\newcommand{\img}[1]{\ten{I}_{#1}} % its tensor dont ccahgen it!!
\newcommand{\imga}{\img{A}}
\newcommand{\imgaaug}{\img{\hat{A}}}
\newcommand{\imgb}{\img{B}}

\newcommand{\dimg}[1]{\ten{D}_{#1}}
\newcommand{\dimga}{\dimg{A}}
\newcommand{\dimgb}{\dimg{B}}

\newcommand{\kp}[1]{\vc{k}_{#1}}
\newcommand{\kpi}[2]{\vc{k}_{#1}^{#2}}
\newcommand{\kpf}[1]{\vc{k}_{#1} = (x_{#1}, y_{#1})}
\newcommand{\kpfi}[2]{\vc{k}_{#1}^{#2} = (x_{#1}^{#2}, y_{#1}^{#2})}

\newcommand{\desc}[1]{\vc{d}_{#1}}

\newcommand{\CCL}{CCL} % In case we want to change how we call it

%%%%%%%%%%%%%%%%%%%%%%%%%%%%%%%%%%%%%%%%%%%%%%%%%%%%%%%%%%%%%%%%
%% Special Commands %%%%%%%%%%%%%%%%%%%%%%%%%%%%%%%%%%%%%%%%%%%%
%%%%%%%%%%%%%%%%%%%%%%%%%%%%%%%%%%%%%%%%%%%%%%%%%%%%%%%%%%%%%%%%

\newcommand{\T}[0]{\mathrm{T}}
\newcommand{\dist}[2]{D_{#2}\left(#1\right)}
\newcommand{\loss}[2]{\mathcal{L}_{#1}^{#2}}

%%%%%%%%%%%%%%%%%%%%%%%%%%%%%%
% Metrics
%%%%%%%%%%%%%%%%%%%%%%%%%%%%%%%
\renewcommand{\mr}{\mathrm{MR}}
\DeclareMathOperator{\mrr}{\mathrm{MRR}}
\newcommand{\hitsat}[1]{\mathrm{HITS@}#1}
\newcommand{\pckat}[1]{\mathrm{PCK@}#1}
\newcommand{\hitsatbr}[1]{\mathrm{H@}#1}
\newcommand{\Set}[1]{\mathcal{#1}}  % set
\newcommand{\cset}[2]{\Set{C}_{#1}^{#2}} % score function with model name

%%%%%%%%%%%%%%%%%%%%%%%%%%%%%%%%%%%%%%%%%%%%%%%%%%%%%%%%%%%%%%%%%%%%%%%%
% Reference short cuts
%%%%%%%%%%%%%%%%%%%%%%%%%%%%%%%%%%%%%%%%%%%%%%%%%%%%%%%%%%%%%%%%%%%%%%%%
\ifdefined\briefref
    \newcommand{\figlower}{fig.}
    \newcommand{\figupper}{Fig.}
    \newcommand{\figlowerp}{fig.}
    \newcommand{\figupperp}{Fig.}
    \newcommand{\seclower}{sec.}
    \newcommand{\secupper}{Sec.}
    \newcommand{\seclowerp}{sec.}
    \newcommand{\secupperp}{Sec.}
    \newcommand{\chaplower}{chap.}
    \newcommand{\chapupper}{Chap.}
    \newcommand{\chaplowerp}{chap.}
    \newcommand{\chapupperp}{Chap.}
    \newcommand{\eqlower}{eq.}
    \newcommand{\equpper}{Eq.}
    \newcommand{\eqlowerp}{eq.}
    \newcommand{\equpperp}{Eq.}
\else
    \newcommand{\figlower}{figure}
    \newcommand{\figupper}{Figure}
    \newcommand{\figlowerp}{figures}
    \newcommand{\figupperp}{Figures}
    \newcommand{\seclower}{section}
    \newcommand{\secupper}{Section}
    \newcommand{\seclowerp}{sections}
    \newcommand{\secupperp}{Sections}
    \newcommand{\chaplower}{chapter}
    \newcommand{\chapupper}{Chapter}
    \newcommand{\chaplowerp}{chapters}
    \newcommand{\chapupperp}{Chapters}
    \newcommand{\eqlower}{equation}
    \newcommand{\equpper}{Equation}
    \newcommand{\eqlowerp}{equations}
    \newcommand{\equpperp}{Equations}
\fi

\newcommand{\figref}[1]{\figlower~\ref{#1}}
\newcommand{\Figref}[1]{\figupper~\ref{#1}}
\newcommand{\figreftwo}[2]{\figlowerp~\ref{#1} and~\ref{#2}}
\newcommand{\Figreftwo}[2]{\figupperp~\ref{#1} and~\ref{#2}}
\newcommand{\figrefthree}[3]{\figlowerp~\ref{#1},~\ref{#2} and~\ref{#3}}
\newcommand{\Figrefthree}[3]{\figupperp~\ref{#1},~\ref{#2} and~\ref{#3}}

\newcommand{\secref}[1]{\seclower~\ref{#1}}
\newcommand{\Secref}[1]{\secupper~\ref{#1}}
\newcommand{\secreftwo}[2]{\seclowerp~\ref{#1} and~\ref{#2}}
\newcommand{\Secreftwo}[2]{\secupperp~\ref{#1} and~\ref{#2}}
\newcommand{\secrefthree}[3]{\seclowerp~\ref{#1},~\ref{#2} and~\ref{#3}}
\newcommand{\Secrefthree}[3]{\secupperp~\ref{#1},~\ref{#2} and~\ref{#3}}

\newcommand{\chapref}[1]{\chaplower~\ref{#1}}
\newcommand{\Chapref}[1]{\chapupper~\ref{#1}}
\newcommand{\chapreftwo}[2]{\chaplowerp~\ref{#1} and~\ref{#2}}
\newcommand{\Chapreftwo}[2]{\chapupperp~\ref{#1} and~\ref{#2}}
\newcommand{\chaprefthree}[3]{\chaplowerp~\ref{#1},~\ref{#2} and~\ref{#3}}
\newcommand{\Chaprefthree}[3]{\chapupperp~\ref{#1},~\ref{#2} and~\ref{#3}}

\let\oldeqref\eqref
\renewcommand{\eqref}[1]{\eqlower~\oldeqref{#1}}  % overwrites the (1)-eqref style
\newcommand{\Eqref}[1]{\equpper~\oldeqref{#1}}
\newcommand{\eqreftwo}[2]{\eqlowerp~\oldeqref{#1} and~\oldeqref{#2}}
\newcommand{\Eqreftwo}[2]{\equpperp~\oldeqref{#1} and~\oldeqref{#2}}
\newcommand{\eqrefthree}[3]{\eqlowerp~\oldeqref{#1},~\oldeqref{#2} and~\oldeqref{#3}}
\newcommand{\Eqrefthree}[3]{\equpperp~\oldeqref{#1},~\oldeqref{#2} and~\oldeqref{#3}}

\newcommand{\expref}[1]{experiment~\ref{#1}}
\newcommand{\Expref}[1]{Experiment~\ref{#1}}
\newcommand{\expreftwo}[2]{experiment \ref{#1} and~\ref{#2}}
\newcommand{\Expreftwo}[2]{Experiment \ref{#1} and~\ref{#2}}
\newcommand{\exprefthree}[3]{experiment \ref{#1},~\ref{#2} and~\ref{#3}}
\newcommand{\Exprefthree}[3]{Experiment \ref{#1},~\ref{#2} and~\ref{#3}}



\newcommand{\appxref}[1]{appendx~\ref{#1}}
\newcommand{\Appxref}[1]{Appendix~\ref{#1}}

\newcommand{\supplref}[1]{supplementary materials~\ref{#1}}


\newcommand{\algoref}[1]{algorithm~\ref{#1}}
\newcommand{\Algoref}[1]{Algorithm~\ref{#1}}
\newcommand{\algoreftwo}[2]{algorithms~\ref{#1} and~\ref{#2}}
\newcommand{\Algoreftwo}[2]{Algorithms~\ref{#1} and~\ref{#2}}
\newcommand{\algorefthree}[3]{algorithms~\ref{#1},~\ref{#2} and~\ref{#3}}
\newcommand{\Algorefthree}[3]{Algorithms~\ref{#1},~\ref{#2} and~\ref{#3}}

\newcommand{\tabref}[1]{table~\ref{#1}}
\newcommand{\Tabref}[1]{Table~\ref{#1}}
\newcommand{\tabreftwo}[2]{tables~\ref{#1} and~\ref{#2}}
\newcommand{\Tabreftwo}[2]{Tables~\ref{#1} and~\ref{#2}}
\newcommand{\tabrefthree}[3]{tables~\ref{#1},~\ref{#2} and~\ref{#3}}
\newcommand{\Tabrefthree}[3]{Tables~\ref{#1},~\ref{#2} and~\ref{#3}}
\newcommand{\tabreffour}[4]{tables~\ref{#1},~\ref{#2},~\ref{#3}, and~\ref{#4}}
\newcommand{\Tabreffour}[4]{Tables~\ref{#1},~\ref{#2},~\ref{#3}, and~\ref{#4}}


\newcommand{\lstref}[1]{listing~\ref{#1}}
\newcommand{\Lstref}[1]{Listing~\ref{#1}}
\newcommand{\lstreftwo}[2]{listings~\ref{#1} and~\ref{#2}}
\newcommand{\Lstreftwo}[2]{Listings~\ref{#1} and~\ref{#2}}
\newcommand{\lstrefthree}[3]{listings~\ref{#1},~\ref{#2} and~\ref{#3}}
\newcommand{\Lstrefthree}[3]{Listings~\ref{#1},~\ref{#2} and~\ref{#3}}

\newcommand{\hyporef}[1]{hypothesis~\ref{#1}}
\newcommand{\Hyporef}[1]{Hypothesis~\ref{#1}}
\newcommand{\hyporeftwo}[2]{hypotheses~\ref{#1} and~\ref{#2}}
\newcommand{\Hyporeftwo}[2]{Hypotheses~\ref{#1} and~\ref{#2}}
\newcommand{\hyporefthree}[3]{hypotheses~\ref{#1},~\ref{#2} and~\ref{#3}}
\newcommand{\Hyporefthree}[3]{Hypotheses~\ref{#1},~\ref{#2} and~\ref{#3}}

\newcommand{\lemref}[1]{lemma~\ref{#1}}
\newcommand{\Lemref}[1]{Lemma~\ref{#1}}
\newcommand{\lemreftwo}[2]{lemma~\ref{#1} and~\ref{#2}}
\newcommand{\Lemreftwo}[2]{Lemma~\ref{#1} and~\ref{#2}}
\newcommand{\lemrefthree}[3]{lemma~\ref{#1},~\ref{#2} and~\ref{#3}}
\newcommand{\Lemrefthree}[3]{Lemma~\ref{#1},~\ref{#2} and~\ref{#3}}

\newcommand{\thmref}[1]{theorem~\ref{#1}}
\newcommand{\Thmref}[1]{Theorem~\ref{#1}}
\newcommand{\thmreftwo}[2]{theorem~\ref{#1} and~\ref{#2}}
\newcommand{\Thmreftwo}[2]{Theorem~\ref{#1} and~\ref{#2}}
\newcommand{\thmrefthree}[3]{theorem~\ref{#1},~\ref{#2} and~\ref{#3}}
\newcommand{\Thmrefthree}[3]{Theorem~\ref{#1},~\ref{#2} and \ref{#3}}


