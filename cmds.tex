% Acro related
\newcommand{\AC}[1]{\ac[long-format=\MakeUppercase]{#1}}
%\newcommand{\ACP}[1]{\acp[long-format=\MakeUppercase]{#1}}

% General utils
\newcommand\todo[1]{\textcolor{red}{TODO: #1}} % Marks the arg red and prepends TODO

% math abuse
\newcommand{\arrowright}{$\rightarrow$}

%%%%%%%%%%%%%%%%%%%%%%%%%%%%%%%%%%%%%%%%%%%%%%%%%%%%%%%%%%%%%%%%%%%%%%%%
% Math
%%%%%%%%%%%%%%%%%%%%%%%%%%%%%%%%%%%%%%%%%%%%%%%%%%%%%%%%%%%%%%%%%%%%%%%%
%\renewcommand{\uv}[1]{\hat{\veg{#1}}} % unit vectors
\newcommand{\sbv}[1]{\uv{e}_{#1}} % standard basis vectors
\renewcommand{\vec}[1]{\mathsfbfit{#1}} %n-dimensional vector

%\renewcommand{\D}{\mathrm{D}} % 
\newcommand{\card}[1]{|#1|} % Cardinality of set
\newcommand{\pnorm}[2]{\left\lVert#1\right\rVert_{#2}} % norm

%\renewcommand{\Set}[1]{\mathcal{#1}}  % set
%\renewcommand{\Graph}[1]{\mathcal{#1}} % graph
%\renewcommand{\GraphSub}[1]{\mathcal{\hat{#1}}}  % graph with args
%\renewcommand{\GraphDef}[1]{\Graph{#1}=(V,E)}  % standard graph def

\newcommand{\bspc}[1]{\{0,1\}^{#1}}
\newcommand{\eye}[1]{\mat{I}_{#1}}
\newcommand{\zeros}[2]{\mat{0}_{#1\times #2}}
%\newcommand{\ones}[2]{\mat{1}_{#1\times #2}}
\newcommand{\ones}[1]{\mathds{1}_{#1}}
\newcommand{\stack}[2]{[#1;#2]}
\newcommand{\iprod}[2]{\langle #1, #2 \rangle}

%\DeclareMathOperator{\tovec}{\mathrm{vec}}
\DeclareMathOperator{\todiag}{\mathrm{diag}}
%\DeclareMathOperator{\R}{\mathbb{R}}
%\newcommand{\T}[0]{\mathrm{T}} 
%\newcommand{\todiag}{\mathrm{diag}}

\newcommand{\spc}[2]{#1^{#2}}
\newcommand{\rspc}[1]{\mathbb{R}^#1}

%\newcommand{\T}{^\operatorname{T}}
\newcommand{\T}{^\intercal}
\DeclareMathOperator{\Tr}{Tr}

% probaiblity stuff
\newcommand{\pd}[2]{p_{#2}(#1)}
\newcommand{\sdf}[2]{#1 \sim #2}
\newcommand{\seq}[3]{#1_{#2:#3}}

\newcommand{\veg}[1]{\mathbfit{#1}}     % geometrical / physical vectors
\newcommand{\vet}[1]{\bm{\mathfrak{#1}}}    % I used it for physical vectors with time-dependency
\newcommand{\rveg}[1]{\bm{ \mathfrak{#1}}}    % random vector
\newcommand{\mat}[1]{\mathsfbfit{#1}} % n x n-dimensional matrix
\newcommand{\rmat}[1]{\mathsfbfit{#1}} % random n x n-dimensional matrix
\newcommand{\ten}[1]{\mathsfbf{#1}} % tensor
\newcommand{\rten}[1]{\mathsfbf{#1}} % random tensor
\newcommand{\uv}[1]{\hat{\veg{#1}}} % unit vectors
\renewcommand{\vec}[1]{\mathsfbfit{#1}} %n-dimensional vector
\newcommand{\op}[1]{\mathcal{#1}} % operator that takes a scalar-valued function
\newcommand{\vecop}[1]{\mathbcal{#1}} % operator that takes a vector-valued function
\newcommand{\mel}[1]{\mathsfit{#1}} 
\newcommand{\vel}[1]{\mathsfit{#1}}
\newcommand{\matel}[1]{\begin{bmatrix} #1 \end{bmatrix}}
\newcommand{\vecel}[1]{\begin{bmatrix} #1 \end{bmatrix}}
\newcommand{\D}{\mathrm{D}}
\newcommand{\inv}[1]{#1^{-1}}

\newcommand{\for}{\quad\mathrm{for}\quad}
\newcommand{\whr}{\quad\mathrm{where}\quad}
\newcommand{\mand}{\quad\mathrm{and}\quad}

\newcommand{\nfjacobian}[3]{
J_{#1}(\vec{u}) = 
\begin{pmatrix}
	\frac{\partial #1_1}{\partial #2_1} & \cdots & \frac{\partial #1_1}{\partial #2_#3} \\
	\vdots  & \ddots & \vdots  \\
	\frac{\partial #1_#3}{\partial #2_1} & \cdots & \frac{\partial #1_#3}{\partial #2_#3}
\end{pmatrix}
}

% complexity
\newcommand{\bigo}[1]{\mathcal{O}(#1)}

% Absolute and Norm
\DeclarePairedDelimiter\abs{\lvert}{\rvert}%
\DeclarePairedDelimiter\norm{\lVert}{\rVert}%

% Swap the definition of \abs* and \norm*, so that \abs
% and \norm resizes the size of the brackets, and the 
% starred version does not.
\makeatletter
\let\oldabs\abs
\def\abs{\@ifstar{\oldabs}{\oldabs*}}
%
\let\oldnorm\norm
\def\norm{\@ifstar{\oldnorm}{\oldnorm*}}
\makeatother